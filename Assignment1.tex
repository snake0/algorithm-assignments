\documentclass{article}
\usepackage{times}
\usepackage{a4wide}
\usepackage{latexsym}
\usepackage{mathrsfs}
\usepackage{amsmath}
\usepackage{amssymb}
\usepackage{amsthm}
\usepackage{ifthen}
\usepackage{stmaryrd}
\usepackage{algorithm}
%\usepackage{algorithmic}
\usepackage{algpseudocode}
\usepackage{graphics}
\usepackage{epsfig}
%\usepackage{ramacros}

\addtolength{\textwidth}{20mm}
\addtolength{\oddsidemargin}{-10mm}
\addtolength{\textheight}{18mm}
\addtolength{\topmargin}{-9mm}

\newcounter{exercise}
\setcounter{exercise}{0}
\newcommand{\exercise}{
        \addtocounter{exercise}{1}
        \vspace{0.2in}
        \noindent
        {\bf \theexercise .}
        }

\newcommand{\REMARK}[1]{}


\newcommand{\NEWPART}{\vspace{.1in}
              \noindent}

\newcommand{\<}{
    \langle}

\renewcommand{\>}{
    \rangle}

\newcommand{\ceil}[1]{\left\lceil #1 \right\rceil}

\pagestyle{plain} \pagenumbering{arabic}

\title{{\bf Assignment 1} \\ {\large Deadline: }}

\author{}
\date{}

\begin{document}
\maketitle

%{\large \noindent
%\begin{tabular}{lcl}
%{Jan 17 2007}.
%\end{tabular}
%}

{\large





\begin{exercise}
Prove the K\"onig theorem: Let $G$ be bipartite, then cardinality of maximum matching = cardinality of minimum vertex cover.
\end{exercise}

\begin{exercise}
Consider the algorithm \textbf{Negative-Dijkstra} for computing shortest paths through graphs with negative edge weights (but without negative cycles)
\begin{algorithm}[htb]
\caption{Algorithm 1: Negative-Dijkstra(G,s)}
\begin{algorithmic}[1]
\State $w^*\leftarrow$ minimum edge weight in $G$;
\For{$e\in E(G)$}
\State $w'(e)\leftarrow w(e)-w^*$
\EndFor
\State $T\leftarrow$\textbf{Dijkstra}$(G',s)$; \\
\Return weights of $T$ in the original $G$;
\end{algorithmic}
\end{algorithm}
Note that \textbf{Negative-Dijkstra} shifts all edge weights to be non-negative(by shifting all edge weights by the smallest original value) and runs in $O(m+\log n)$ time.\\
Prove or Disprove: \textbf{Negative-Dijkstra} computes single-source shortest paths correctly in graphs with negative edge weights. To prove the algorithm correct, show that for all $u\in V$ the shortest $s-u$ path in the original graph is in $T$. To disprove, exhibit a graph with negative edges, with no negative cycles where \textbf{Negative-Dijkstra} outputs the wrong "shortest" paths, and explain why the algorithm fails.
\end{exercise}

\begin{exercise}
Consider a weighted, directed graph $G$ with $n$ vertices and $m$ edges that have integer weights. A graph walk is a sequence of
not-necessarily-distinct vertices $v_1,v_2,\ldots,v_k$ such that each pair of consecutive vertices $v_i,v_{i+1}$ are connected by an edge. This is similar to a path, except a walk can have repeated vertices and edges. The length of a walk in a weighted graph is the sum of the weights of the edges in the walk. Let $s,t$ be given vertices in the graph, and $L$ be a positive integer. We are interested counting the number of walks from $s$ to $t$ of length exactly $L$.
\begin{itemize}
\item Assume all the edge weights are positive. Describe an algorithm that computes the number of graph walks from $s$ to $t$ of length exactly $L$ in $O((n+m)L)$ time. Prove the correctness and analyze the running time
\item Now assume all the edge weights are non-negative(but they can be 0), but there are no cycles consisting entirely of zero-weight edges. That is, for any cycle in the graph, at least one edge has a positive weight.\\
Describe an algorithm that computes the number of graph walks from $s$ to $t$ of length exactly $L$ in $O((n+m)L)$ time. Prove correctness and analyze running time.
\end{itemize}
\end{exercise}
\newpage
\begin{exercise}
The diameter of a connected, undirected graph $G=(V,E)$ is the length (in number of edges) of the longest shortest path between two nodes. Show that is the diameter of a graph is $d$ then there is some set $S\subseteq V$ with $|S|\leq n/(d-1)$ such that removing the vertices in $S$ from the graph would break it into several disconnected pieces.

\end{exercise}

\begin{exercise}
Let $G$ be a $n$ vertices graph. Show that if every vertex in $G$ has degree at least $n/2$, then $G$ contains a Hamiltonian path.
\end{exercise}

\begin{exercise}
Show how to find a minimal cut of a graph (not only the cost of minimum cut, but also the set of edges in the cut).
\end{exercise}

\begin{exercise}
Let $G(V,E)$ be a connected undirected graph with a weight $w(e)>0$ for each edge $e\in E$. For any path $P_{u,v}=<u,v_1,v_2,\ldots,v_r,v>$ between two vertices $u$ and $v$ in $G$,let $\beta(P_{u,v})$ denote the maximum weight of an edge in $P_{u,v}$. We refer to $\beta(P_{u,v})$ as the \textbf{bottleneck weight} of $P_{u,v}$. Define
\begin{displaymath}
\beta^*(u,v)=\min\{\beta(P_{u,v}):P_{u,v}\text{ is a path between $u$ and $v$}\}.
\end{displaymath}
Give a polynomial algorithm to find $\beta^*(u,v)$ for each pair of vertices $u$ and $v$ in $V$ and a proof of the correctness of the algorithm.
\end{exercise}

\begin{exercise}
Let $G=(V,E)$ be a directed graph. Give a linear-time algorithm that given $G$, a node $s\in V$ and an integer $k$ decides whether there is a walk in $G$ starting at $s$ that visits at least $k$ distinct nodes.
\end{exercise}

\begin{exercise}
\textbf{Minimum Bottleneck Spanning Tree}: Given a connected graph $ G $ with positive edge costs, find a
spanning tree that {minimizes the most expensive edge}.
\end{exercise}

\end{document}
