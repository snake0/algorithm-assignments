\documentclass{article}
\usepackage{times}
\usepackage{a4wide}
\usepackage{latexsym}
\usepackage{mathrsfs}
\usepackage{amsmath}
\usepackage{amssymb}
\usepackage{amsthm}
\usepackage{ifthen}
\usepackage{stmaryrd}
\usepackage{algorithm}
%\usepackage{algorithmic}
\usepackage{algpseudocode}
\usepackage{graphics}
\usepackage{epsfig}
%\usepackage{ramacros}

\addtolength{\textwidth}{20mm}
\addtolength{\oddsidemargin}{-10mm}
\addtolength{\textheight}{18mm}
\addtolength{\topmargin}{-9mm}

\newcounter{exercise}
\setcounter{exercise}{0}
\newcommand{\exercise}{
        \addtocounter{exercise}{1}
        \vspace{0.2in}
        \noindent
        {\bf \theexercise .}
        }

\newcommand{\REMARK}[1]{}


\newcommand{\NEWPART}{\vspace{.1in}
              \noindent}

\newcommand{\<}{
    \langle}

\renewcommand{\>}{
    \rangle}

\newcommand{\ceil}[1]{\left\lceil #1 \right\rceil}

\pagestyle{plain} \pagenumbering{arabic}

\title{{\bf Assignment 3} \\ {\large ID: 120037910002 } {\large Name: Xingguo Jia } {\large Email: jiaxg1998@sjtu.edu.cn}}

\author{}
\date{}

\begin{document}
\maketitle

%{\large \noindent
%\begin{tabular}{lcl}
%{Jan 17 2007}.
%\end{tabular}
%}

{\large





\begin{exercise}
Prove or disprove the following statement. If all capacities in a network are distinct,
 then there exists a unique flow function that gives the maximum flow.
\end{exercise}
\begin{itemize}
    \item 
    \item 
\end{itemize}



\begin{exercise}
An edge of a flow network is called \textbf{critical} if decreasing the capacity of this edge results in a decrease in the maximum flow value. Present an efficient algorithm that, given an $s$-$t$ network $G$ finds any critical edge in a network(assuming one exists).
\end{exercise}
\begin{itemize}
    \item 
    \item 
\end{itemize}



\begin{exercise}
Let $G=(V,E)$ be an undirected weighted graph with two distinguished vertices $s,t\in V$. Give an efficient algorithm to find a minimum weight cut that separates $s$ from $t$.
\end{exercise}
\begin{itemize}
    \item 
    \item 
\end{itemize}



\begin{exercise}
You are given a matrix with fractional elements between $0$ and $1$. The sum of all numbers in each row and in each column is integer. Prove that we can always round each element to $0$ or $1$ so that the sum of each row and each column remains unchanged and design a polynomial time algorithm to find such a rounding result.
\end{exercise}
\begin{itemize}
    \item 
    \item 
\end{itemize}



\begin{exercise}
Suppose that, in addition to edge capacities, a flow network has \textbf{vertex capacities}.
That is each vertex has a limit on how much flow can pass though. Show how to transform a flow network $G=(V,E)$ with vertex capacities into an equivalent flow network $G'=(V',E')$ without vertex capacities, such that a maximum flow in $G'$ has the same value as a maximum flow in $G$. How many vertices and edges does $G'$ have?
\end{exercise}
\begin{itemize}
    \item 
    \item 
\end{itemize}




\begin{exercise}
Consider a bipartite graph $G=(X\cup Y,E)$ with parts $X$ and $Y$. Each part contains $2k$ vertices (i.e. $|X|=|Y|=2k$). Suppose that $deg(u)\geq k$ for every $u\in X\cup Y$. Prove that $G$ has a perfect matching.
\end{exercise}
\begin{itemize}
    \item 
    \item 
\end{itemize}




\begin{exercise}
You are designing a experiment in which you want to measure certain properties $p_1,\ldots ,p_n$ of a yeast culture. You have a set of tools $t_1,\ldots ,t_m$ that can each measure a subset $S_i$ of the properties. For example, tool $t_i$ measures $S_i$ may equal $\{p_7,p_8\}$. To be sure that your results are not due to noise or other artifact, you must measure every property at least $k$ times using $k$ different tools.
\begin{itemize}
\item Give a polynomial-time algorithm that decides whether the tools you have are sufficient to measure the desired properties the desired number of times.
\item Suppose each tool $t_i$ comes from manufacturer $M_i$ and we have the additional constraint that the tools to test any property $p_i$ can't all come from the same manufacturer. Give a polynomial-time algorithm to solve this problem.
\end{itemize}
\end{exercise}
\begin{itemize}
    \item 
    \item 
\end{itemize}




\begin{exercise}
Consider a flow network $G=(V,E)$ with positive edge capacities $\{c(e)\}$. Let $f:E\rightarrow \mathbb{R}_{\geq 0}$ be a maximum flow in $G$, and $G_f$ be the residual graph. Denote by $S$ the set of nodes reachable from $s$ in $G_f$ and by $T$ the set of nodes from which $t$ is reachable in $G_f$.  Prove that $V=S\cup T$ if and only if $G$ has a \textbf{unique} $s$-$t$ minimum cut.
\end{exercise}
\begin{itemize}
    \item 
    \item 
\end{itemize}




\end{document}
