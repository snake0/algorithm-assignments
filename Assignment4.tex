\documentclass{article}
\usepackage{times}
\usepackage{a4wide}
\usepackage{latexsym}
\usepackage{mathrsfs}
\usepackage{amsmath}
\usepackage{amssymb}
\usepackage{amsthm}
\usepackage{ifthen}
\usepackage{stmaryrd}
\usepackage{algorithm}
%\usepackage{algorithmic}
\usepackage{algpseudocode}
\usepackage{graphics}
\usepackage{epsfig}
%\usepackage{ramacros}

\addtolength{\textwidth}{20mm}
\addtolength{\oddsidemargin}{-10mm}
\addtolength{\textheight}{18mm}
\addtolength{\topmargin}{-9mm}

\newcounter{exercise}
\setcounter{exercise}{0}
\newcommand{\exercise}{
        \addtocounter{exercise}{1}
        \vspace{0.2in}
        \noindent
        {\bf \theexercise .}
        }

\newcommand{\REMARK}[1]{}


\newcommand{\NEWPART}{\vspace{.1in}
              \noindent}

\newcommand{\<}{
    \langle}

\renewcommand{\>}{
    \rangle}

\newcommand{\ceil}[1]{\left\lceil #1 \right\rceil}

\pagestyle{plain} \pagenumbering{arabic}

\title{{\bf Assignment 4} \\ {\large ID: 120037910002 } {\large Name: Xingguo Jia } {\large Email: jiaxg1998@sjtu.edu.cn}}

\author{}
\date{}

\begin{document}
\maketitle

%{\large \noindent
%\begin{tabular}{lcl}
%{Jan 17 2007}.
%\end{tabular}
%}

{\large

\begin{exercise}
	Transform following problems into linear programming
	
	(1)
		$$\begin{aligned}
		&\text { maximize } & 5x+2y\\
		&\text { subject to } & 0 \le x \le 20 \\
		& & |x - y|\le 5
		\end{aligned}$$
		
	(2)
		$$\begin{aligned}
		&\text { maximize }  & min(x1,x2,x3)\\
		&\text { subject to }  & x1+x2+x3 = 15
		\end{aligned}$$
\end{exercise}

\textbf{Solution:}

	(1)
		$$\begin{aligned}
		&\text { maximize } & 5x+2y\\
		&\text { subject to } & x \le 20 \\
		& & x \ge 0 \\
		& & x - y\le 5 \\
		& & x - y\ge -5 \\
		\end{aligned}$$

	(2)
		$$\begin{aligned}
		&\text { maximize }  & x1\\
		&\text { subject to }  & x1+x2+x3 = 15 \\
		& & x1\leq x2\\
		& & x1\leq x3\\
		\end{aligned}$$

\newpage


\begin{exercise}
	Given a graph $G$, each vertex $v_i$ has a profit $p_i$ and each edge $e_{ij}$ has a cost $c_{ij}$. Define the profit of a cycle is the total profits of all the vertices in the cycle, and the cost of a cycle is the total costs of all the edges in the cycle. We need to find a simple cycle in $G$ which contains a given vertex $v_0$, and maximize the profit of it within the cost bound $B$. Write the linear programming formulation of this problem.
\end{exercise}

\textbf{Solution:}
\begin{itemize}
	\item We have $x_i$ for each $v_i\in V$, if $v_i$ is in the simple cycle, $x_i=1$, else $x_i=0$. We have $y_{ij}$ for each $e_{ij}\in E$, if $e_{ij}$ is in the simple cycle, $y_{ij}=1$, else $y_{ij}=0$.
\end{itemize}

	$$\begin{aligned}
	&\text { maximize }  & \sum_{v_i\in V}{x_i*p_i} && \\
	&\text { subject to }  & x_0 = 1 & & \\
	& & x_i\leq 1,x_i\geq 0 & &v_i\in V\\
	& & y_{ij}\leq 1,y_{ij}\geq 0 & &e_{ij}\in E\\
	& & \sum_{e_{ij}\in E}{y_{ij}*c_{ij}}\leq B & & \\
	& & \sum_{e_{ij}\in E}{y_{ij}}\geq 2x_{i} & & v_i\in V \\
	& & \sum_{e_{ij}\in E}{y_{ij}}\leq 2 & & v_i\in V \\
	& & x_i+x_j\geq 2y_{ij} & & e_{ij}\in E
	\end{aligned}$$
	
\newpage




\begin{exercise}
	Consider the following optimization problem
	$$\begin{aligned}
		&\text { minimize } \quad f_{0}(\mathbf{x})\\
		&\text { subject to } f_{i}(\mathbf{x}) \leq 0, \quad i=1, \ldots, m
	\end{aligned}$$
	with variable $\mathbf{x}=\left(x_{1}, \ldots, x_{n}\right) \in \mathbf{R}^{n} .$ Note that the program may not be linear. The Lagrangian
	$L: \mathbf{R}^{n} \times \mathbf{R}^{m} \rightarrow \mathbf{R}$ associated with the program is defined as
	$$
	L(\mathbf{x}, \lambda)=f_{0}(\mathbf{x})+\sum_{i=1}^{m} \lambda_{i} f_{i}(\mathbf{x})
	$$
	where $\lambda=\left(\lambda_{1}, \ldots, \lambda_{m}\right) \in \mathbf{R}^{m}$
	
	Define the Lagrange dual function $g: \mathbf{R}^{m} \rightarrow \mathbf{R}$ as the minimum value of the Lagrangian over $x$ :
	for $\lambda \in \mathbf{R}^{m}$,
	$$
	g(\lambda)=\inf _{\mathbf{x}} L(\mathbf{x}, \lambda)
	$$
	We write $\lambda \geq 0$ if $\lambda_{i} \geq 0$ for all $1 \leq i \leq m$ and let $p^{*}$ be the optimal value of original program.
	
	Show that:
	$$
	g(\lambda) \leq p^{*} , \text{ for every } \lambda \geq 0
	$$
\end{exercise}
\textbf{Solution:}
\begin{itemize}
    \item Since $\lambda_{i}\geq 0$ and $f_{i}\leq 0$, we have 
		$$\sum_{i=1}^{m}{\lambda_{i}f_i(x)}\leq 0$$.
    \item Let $f_0(x^*)=p^*$, we have 
		$$ g(\lambda)=\inf _{\mathbf{x}} L(\mathbf{x}, \lambda) \leq \mathbf{x^*} L(\mathbf{x^*}, \lambda) = f_0(x^*) + \sum_{i=1}^{m}{\lambda_{i}f_i(x)}\leq p^* $$
\end{itemize}
\newpage



	
\begin{exercise}
	Consider the following optimization problem
	$$\begin{aligned}
	&\text { maximize } g(\lambda)\\
	&\text { subject to } \lambda \ge 0
	\end{aligned}$$
	Show that if the program in previous exercise is a linear program, then this program is its dual program.
\end{exercise}

\textbf{Solution:}
\begin{itemize}
    \item We have 
		$$ L(\mathbf{x}, \lambda) = c^{T}\mathbf{x}+\lambda_{1}^{T}(b-Ax)+\lambda_{2}^{T}*(-\mathbf{x}) $$
		$$ = (c^{T}-\lambda_{1}^{T}A-\lambda_{2}^{T})\mathbf{x}+\lambda_{1}^{T}b$$
		$$ =(c-A^{T}\lambda_{1}-\lambda_{2})^{T}*\mathbf{x}+\lambda_{1}b^{T} $$
    \item Then 
		\begin{equation} g(\lambda)=\inf _{\mathbf{x}} L(\mathbf{x}, \lambda) =
		\begin{cases} 
			\lambda_{1}b^{T}, & c-A^{T}\lambda_{1}-\lambda_{2}=0 \\ 
			-\infty, &  otherwise
		\end{cases} 
		\end{equation}
		\item Thus, the above problem can be transformed into 
		$$\begin{aligned}
		&\text { maximize } &\lambda_{1}b^{T}\\
		&\text { subject to }& c-A^{T}\lambda_{1}-\lambda_{2}=0\\
		& & \lambda_{1},\lambda_{2} \geq 0
		\end{aligned}$$
		\item which can be transformed into
		$$\begin{aligned}
			&\text { maximize } &\lambda_{1}b^{T}\\
			&\text { subject to }& c-A^{T}\lambda_{1}\geq 0\\
			& & \lambda_{1}\geq 0
			\end{aligned}$$
\end{itemize}
\newpage




\begin{exercise}
	Prove the complementary slackness property of linear programs.
\end{exercise}

\begin{proof}
    The complementary slackness theorem is: Assume LP problem (P) has a solution $x^*$ and its dual problem (D) has a solution $y^*$.
		\begin{itemize}
			\item (1) If $x^{*}_j > 0$, then the $j$-th constraint in (D) is binding.
			\item (2) If the $j$-th constraint in (D) is not binding, then $x^{*}_j = 0$.
			\item (3) If $y^{*}_j > 0$, then the $i$-th constraint in (P) is binding.
			\item (4) If the $i$-th constraint in (P) is not binding, then $y^{*}_j = 0$.
			\item Suppose problem P is

			$$\begin{aligned}
			&\text { maximize } &\mathbf{c}^{T}\mathbf{x}\\
			&\text { subject to }& \mathbf{Ax}\leq \mathbf{b}\\
			& & \mathbf{x\geq 0}
			\end{aligned}$$

			\item and its dual problem D
			$$\begin{aligned}
			&\text { minimize } &\mathbf{y}^{T}\mathbf{b}\\
			&\text { subject to }& \mathbf{y}^{T}\mathbf{A}\geq \mathbf{c}^T\\
			& & \mathbf{y\geq 0}
			\end{aligned}$$

			\item Suppose $\mathbf{x^*}$ and $\mathbf{y^*}$ are the optimum solution of problem P and D, respectively. Then according to problem D, we have
			$$
			\mathbf{y*}^{T}\mathbf{A}\geq \mathbf{c}^T
			$$
			\item Since $\mathbf{x^*\geq 0}$, multiply $x^*$, we have
			$$
			\mathbf{y*}^{T}\mathbf{Ax^*}\geq \mathbf{c}^T\mathbf{x^*}
			$$
			\item According to problem P, we have
			$$
			\mathbf{y*}^{T}\mathbf{b} \geq \mathbf{y*}^{T}\mathbf{Ax^*}
			$$
			\item Then 
			$$
			\mathbf{y*}^{T}\mathbf{b} \geq \mathbf{y*}^{T}\mathbf{Ax^*}\geq \mathbf{c}^T\mathbf{x^*}
			$$
			\item According to the Strong Duality Theorem, we have
			$$
			\mathbf{y*}^{T}\mathbf{b} = \mathbf{y*}^{T}\mathbf{Ax^*} = \mathbf{c}^T\mathbf{x^*}
			$$
			\item which implies
			$$
			\mathbf{y*}^{T}\mathbf{(Ax^*-b)}=0,(\mathbf{y*}^{T}\mathbf{A}-\mathbf{c}^T)\mathbf{x^*}=0
			$$
			\item The former inequality implies (1),(2) of the complementary slackness property, the latter inequality implies (3),(4) of the complementary slackness property.
		\end{itemize}
\end{proof}
\newpage




\begin{exercise}
	Write the dual problem of:
	$$\begin{aligned}
	&\text { maximize } & c^Tx\\
	&\text { subject to } & A_1x\le b_1 \\
	& &A_2x \ge b_2 \\
	& &A_3x = b_3
	\end{aligned}$$
\end{exercise}

\textbf{Solution:}
\begin{itemize}
    \item The problem can be rewritten as
		$$\begin{aligned}
			&\text { maximize } & c^T(x_1-x_2)\\
			&\text { subject to } & A_1(x_1-x_2)\le b_1 \\
			& &-A_2(x_1-x_2) \le -b_2 \\
			& &A_3(x_1-x_2) \le b_3 \\ 
			& &-A_3(x_1-x_2) \le -b_3 \\ 
			& & x_1,x_2\geq 0
			\end{aligned}$$
    \item Then its dual is 
		$$\begin{aligned}
			&\text {minimize} & b_{1}^{T}y_1-b_{2}^Ty_2 + b_{3}^{T}y_3\\
			&\text {subject to } & A_{1}^{T}y_1-A_{2}^{T}y_2+A_{3}^{T}y_3-A_{3}^{T}y_4=c^T\\
			& & y_1,y_2,y_3,y_4\geq 0
			\end{aligned}$$
\end{itemize}
\newpage


	


\begin{exercise}
	Prove the König theorem: Let G be bipartite, then cardinality of maximum matching = cardinality of minimum vertex cover.
\end{exercise}

\textbf{Solution:}
\begin{itemize}
    \item The maximum matching problem can be written as the following linear program
		$$\begin{aligned}
			&\text { maximize } &\sum_{e\in E}{x_e}&&\\
			&\text { subject to }& \sum_{e=(u,v)\in E}x_{e}\leq 1, &&v\in V\\
			& & {x_e\geq 0,}&&e\in E
			\end{aligned}$$
    \item Consider its dual LP
		$$\begin{aligned}
			&\text { minimize } &\sum_{v\in V}{y_v}&&\\
			&\text { subject to }& y_u+y_v\geq 1, &&e=(u,v)\in E\\
			& & {y_v\geq 0,}&&v\in V
			\end{aligned}$$
			\item Which means the minimum vertex cover problem. According to the Strong Duality Theorem, then $\textbf { minimize } \sum_{v\in V}{y_v} = \textbf { maximize } \sum_{e\in E}{x_e}$, we have cardinality of maximum matching = cardinality of minimum vertex cover.
\end{itemize}
\newpage



\begin{exercise}
	Show that the dual of the dual of a linear program is the primal linear program.
\end{exercise}

\textbf{Solution:}
\begin{itemize}
    \item Suppose the problem P is
		$$\begin{aligned}
			&\text { maximize } &\mathbf{c}^{T}\mathbf{x}\\
			&\text { subject to }& \mathbf{Ax}\leq \mathbf{b}\\
			& & \mathbf{x\geq 0}
			\end{aligned}$$
    \item Then its dual problem D is
		$$\begin{aligned}
			&\text { minimize } &\mathbf{y}^{T}\mathbf{b}\\
			&\text { subject to }& \mathbf{y}^{T}\mathbf{A}\geq \mathbf{c}^T\\
			& & \mathbf{y\geq 0}
			\end{aligned}$$
		\item Transpose the above formula
		$$\begin{aligned}
			&\text { minimize } &\mathbf{b}^{T}\mathbf{y}\\
			&\text { subject to }& \mathbf{A}^{T}\mathbf{y}\geq \mathbf{c}\\
			& & \mathbf{y\geq 0}
			\end{aligned}$$
		\item which can be rewritten as 
		$$\begin{aligned}
			&\text { maximize } &-\mathbf{b}^{T}\mathbf{y}\\
			&\text { subject to }& -\mathbf{A}^{T}\mathbf{y}\leq -\mathbf{c}\\
			& & \mathbf{y\geq 0}
			\end{aligned}$$
		\item Then the dual problem of the above problem is
		$$\begin{aligned}
			&\text { minimize } &-\mathbf{c}^{T}\mathbf{z}\\
			&\text { subject to }& -\mathbf{A}\mathbf{z}\geq -\mathbf{b}\\
			& & \mathbf{z\geq 0}
			\end{aligned}$$
		\item Which can be rewritten as
		$$\begin{aligned}
			&\text { maximize } & \mathbf{c}^{T}\mathbf{z}\\
			&\text { subject to }& \mathbf{A}\mathbf{z}\leq \mathbf{b}\\
			& & \mathbf{z\geq 0}
			\end{aligned}$$
			\item Let $z=x$, then the above problem becomes problem P.
\end{itemize}
\newpage


	



\end{document}
