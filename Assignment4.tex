\documentclass{article}
\usepackage{times}
\usepackage{a4wide}
\usepackage{latexsym}
\usepackage{mathrsfs}
\usepackage{amsmath}
\usepackage{amssymb}
\usepackage{amsthm}
\usepackage{ifthen}
\usepackage{stmaryrd}
\usepackage{algorithm}
%\usepackage{algorithmic}
\usepackage{algpseudocode}
\usepackage{graphics}
\usepackage{epsfig}
%\usepackage{ramacros}

\addtolength{\textwidth}{20mm}
\addtolength{\oddsidemargin}{-10mm}
\addtolength{\textheight}{18mm}
\addtolength{\topmargin}{-9mm}

\newcounter{exercise}
\setcounter{exercise}{0}
\newcommand{\exercise}{
        \addtocounter{exercise}{1}
        \vspace{0.2in}
        \noindent
        {\bf \theexercise .}
        }

\newcommand{\REMARK}[1]{}


\newcommand{\NEWPART}{\vspace{.1in}
              \noindent}

\newcommand{\<}{
    \langle}

\renewcommand{\>}{
    \rangle}

\newcommand{\ceil}[1]{\left\lceil #1 \right\rceil}

\pagestyle{plain} \pagenumbering{arabic}

\title{{\bf Assignment 4} \\ {\large ID: 120037910002 } {\large Name: Xingguo Jia } {\large Email: jiaxg1998@sjtu.edu.cn}}

\author{}
\date{}

\begin{document}
\maketitle

%{\large \noindent
%\begin{tabular}{lcl}
%{Jan 17 2007}.
%\end{tabular}
%}

{\large

\begin{exercise}
	Transform following problems into linear programming
	
	(1)
		$$\begin{aligned}
		&\text { maximize } & 5x+2y\\
		&\text { subject to } & 0 \le x \le 20 \\
		& & |x - y|\le 5
		\end{aligned}$$
		
	(2)
		$$\begin{aligned}
		&\text { maximize }  & min(x1,x2,x3)\\
		&\text { subject to }  & x1+x2+x3 = 15
		\end{aligned}$$
\end{exercise}

\textbf{Solution:}
\begin{itemize}
    \item 
    \item 
\end{itemize}
\newpage


\begin{exercise}
	Given a graph $G$, each vertex $v_i$ has a profit $p_i$ and each edge $e_{ij}$ has a cost $c_{ij}$. Define the profit of a cycle is the total profits of all the vertices in the cycle, and the cost of a cycle is the total costs of all the edges in the cycle. We need to find a simple cycle in $G$ which contains a given vertex $v_0$, and maximize the profit of it within the cost bound $B$. Write the linear programming formulation of this problem.
\end{exercise}

\textbf{Solution:}
\begin{itemize}
    \item 
    \item 
\end{itemize}
\newpage




\begin{exercise}
	Consider the following optimization problem
	$$\begin{aligned}
		&\text { minimize } \quad f_{0}(\mathbf{x})\\
		&\text { subject to } f_{i}(\mathbf{x}) \leq 0, \quad i=1, \ldots, m
	\end{aligned}$$
	with variable $\mathbf{x}=\left(x_{1}, \ldots, x_{n}\right) \in \mathbf{R}^{n} .$ Note that the program may not be linear. The Lagrangian
	$L: \mathbf{R}^{n} \times \mathbf{R}^{m} \rightarrow \mathbf{R}$ associated with the program is defined as
	$$
	L(\mathbf{x}, \lambda)=f_{0}(\mathbf{x})+\sum_{i=1}^{m} \lambda_{i} f_{i}(\mathbf{x})
	$$
	where $\lambda=\left(\lambda_{1}, \ldots, \lambda_{m}\right) \in \mathbf{R}^{m}$
	
	Define the Lagrange dual function $g: \mathbf{R}^{m} \rightarrow \mathbf{R}$ as the minimum value of the Lagrangian over $x$ :
	for $\lambda \in \mathbf{R}^{m}$,
	$$
	g(\lambda)=\inf _{\mathbf{x}} L(\mathbf{x}, \lambda)
	$$
	We write $\lambda \geq 0$ if $\lambda_{i} \geq 0$ for all $1 \leq i \leq m$ and let $p^{*}$ be the optimal value of original program.
	
	Show that:
	$$
	g(\lambda) \leq p^{*} , \text{ for every } \lambda \geq 0
	$$
\end{exercise}
\textbf{Solution:}
\begin{itemize}
    \item 
    \item 
\end{itemize}
\newpage



	
\begin{exercise}
	Consider the following optimization problem
	$$\begin{aligned}
	&\text { maximize } g(\lambda)\\
	&\text { subject to } \lambda \ge 0
	\end{aligned}$$
	Show that if the program in previous exercise is a linear program, then this program is its dual program.
\end{exercise}

\textbf{Solution:}
\begin{itemize}
    \item 
    \item 
\end{itemize}
\newpage




\begin{exercise}
	Prove the complementary slackness property of linear programs.
\end{exercise}

\textbf{Solution:}
\begin{itemize}
    \item 
    \item 
\end{itemize}
\newpage




\begin{exercise}
	Write the dual problem of:
	$$\begin{aligned}
	&\text { maximize } & c^Tx\\
	&\text { subject to } & A_1x\le b_1 \\
	& &A_2x \ge b_2 \\
	& &A_3x = b_3
	\end{aligned}$$
\end{exercise}

\textbf{Solution:}
\begin{itemize}
    \item 
    \item 
\end{itemize}
\newpage


	


\begin{exercise}
	Prove the König theorem: Let G be bipartite, then cardinality of maximum matching = cardinality of minimum vertex cover.
\end{exercise}

\textbf{Solution:}
\begin{itemize}
    \item 
    \item 
\end{itemize}
\newpage



\begin{exercise}
	Show that the dual of the dual of a linear program is the primal linear program.
\end{exercise}

\textbf{Solution:}
\begin{itemize}
    \item 
    \item 
\end{itemize}
\newpage


	



\end{document}
